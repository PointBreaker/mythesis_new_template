%---------------------------------------------------------------------------%
%->> Abstract
%---------------------------------------------------------------------------%
%-
%-> 中文摘要
%-
\begin{abstract}
% 随着无线通信技术的迅猛发展,尤其是在5G和6G技术的推动下,我们进入了一个宽带信号处理日益重要的时代。在这样的背景下,自动调制识别(Automatic Modulation Recognition, AMR)成为通信领域的关键技术,尤其在认知无线电、频谱感知、信号监视等方面发挥着至关重要的作用。然而,随着频谱环境的日益复杂化,传统的AMR技术面临着越来越多的挑战,特别是在低信噪比和多径衰落等恶劣环境下的准确性和鲁棒性问题。这些挑战迫切需要新的技术解决方案。

% 本论文针对这一问题,引入了基于深度学习的方法来提升AMR技术。深度学习,作为一种强大的特征提取和模式识别工具,为AMR领域带来了新的研究方向。本研究主要集中在三个方面:提升低信噪比环境下的调制识别准确率、宽带信号调制识别以及宽带信号的解调技术。
    
% 首先,针对低信噪比环境下的调制识别问题,本研究提出了一种创新的自适应噪声矫正模块。这一模块通过动态调整机制,能够根据环境噪声水平的实时变化自动调节矫正强度,显著提升了系统在复杂噪声背景下的性能。这一技术的引入不仅提高了识别的准确率,而且增强了系统的鲁棒性。其次,本研究在宽带信号调制识别方面取得了显著成果。随着宽带通信技术的发展,对于宽带信号的处理需求日益增长。在这方面,本研究不仅关注于识别信号的调制模式,还包括了确定信号占用的子带位置。这一研究的意义在于,它不仅拓展了AMR的应用范围,还对解决宽带信号处理中的新挑战至关重要。最后,本研究还着眼于宽带信号的调制解调过程。本研究提出了一套综合性的解调框架,能够有效地从宽带信号中提取出有用信息,并进行准确的解调。这包括识别子带位置、调制模式,以及符号长度,最终输出宽带调制信号解调后的比特流。这一研究的意义在于,它不仅提高了调制识别的准确性和效率,还实现了从信号检测到信息恢复的完整流程。

在当前的通信技术飞速发展时代,尤其是随着第五代(5G)和预期中的第六代(6G)无线通信技术的推进,宽带信号处理变得尤为关键。自动调制识别(Automatic Modulation Recognition, AMR)作为一项至关重要的技术,不仅在认知无线电、频谱感知和信号监视等领域中起到了核心作用,而且对于提高通信效率和网络管理的智能化程度具有重大意义。尽管如此,随着频谱环境的不断复杂化,特别是在低信噪比和多径衰落等恶劣条件下,现有的自动调制识别技术面临着严峻挑战,这些挑战包括识别准确性的降低和系统鲁棒性的不足,迫切需要采用新的技术手段来解决。本论文通过引入深度学习技术在自动调制识别中的应用,提出了一系列创新性解决方案,旨在克服现有技术的局限并推动自动调制识别技术的发展。本论文的贡献主要集中在以下三个方面:

1.提高低信噪比环境下的调制识别准确性:本论文提出了一种自适应噪声矫正模块,该模块采用创新的动态调整机制,能够根据环境噪声水平的实时变化自动调整矫正策略,从而显著提升系统在复杂噪声背景下的性能。

2.宽带信号调制识别:本论文在宽带信号调制识别方面取得一定进展,不仅关注于识别信号的调制模式,还创新性地涉及到确定信号所占用的子带位置。这一研究拓展了自动调制识别技术的应用范围,并应对宽带信号处理的新挑战。

3.基于深度学习的宽带信号调制解调:本论文提出了一整套创新的解调框架,能够高效从宽带信号中提取关键信息并进行解调。该框架不仅包括识别子带位置、调制模式和符号长度,还能够输出解调后的比特流,实现了从信号的检测到信息恢复的完整流程,为宽带信号处理提出了一套完整的框架。

综上所述,本论文通过深度学习技术的应用,为解决自动调制识别技术在复杂环境下面临的挑战提供了有效的技术途径,对于推动无线通信技术的进步和提高通信网络的智能化水平具有重要的理论和实践意义。


    \keywords{无线通信、自动调制识别、深度学习}% 中文关键词
\end{abstract}
%-
%-> 英文摘要
%-
\begin{ABSTRACT}
% With the rapid advancement of wireless communication technology, especially driven by 5G and 6G innovations, we have entered an era where wideband signal processing is increasingly crucial. Against this backdrop, Automatic Modulation Recognition (AMR) has become a key technology in the field of communication, particularly vital in cognitive radio, spectrum sensing, and signal monitoring. However, traditional AMR techniques are facing growing challenges, especially in terms of accuracy and robustness in low signal-to-noise ratio (SNR) and multipath fading environments. These challenges urgently require new technological solutions.

% This thesis introduces a deep learning-based approach to enhance AMR technology. As a powerful tool for feature extraction and pattern recognition, deep learning opens new research directions in the field of AMR. This study focuses on three main areas: improving modulation recognition accuracy in low SNR environments, recognizing modulation in wideband signals, and demodulating wideband signals.
    
% Firstly, addressing modulation recognition in low SNR environments, this research proposes an innovative adaptive noise correction module. This module, through a dynamic adjustment mechanism, can automatically adjust correction intensity based on real-time changes in noise levels, significantly enhancing system performance in complex noise backgrounds. This technique not only improves recognition accuracy but also enhances the system's robustness. Secondly, significant progress has been made in wideband signal modulation recognition. With the development of wideband communication technology, there is an increasing need to process wideband signals. This research not only focuses on recognizing modulation patterns but also on identifying the occupied sub-band positions in wideband signals. This aspect of the study is crucial for expanding the application range of AMR and addressing new challenges in wideband signal processing. Lastly, the research focuses on the demodulation process of wideband signals. A comprehensive demodulation framework has been proposed, capable of effectively extracting useful information from wideband signals and accurately demodulating them. This includes identifying sub-band positions, modulation patterns, and symbol lengths, ultimately outputting the bitstream of the demodulated wideband modulation signal. This research is significant as it not only enhances the accuracy and efficiency of modulation recognition but also achieves a complete process from signal detection to information recovery.
In the current era of rapid development in communication technology, especially with the advancement of the fifth-generation (5G) and the anticipated sixth-generation (6G) wireless communication technologies, wideband signal processing has become particularly crucial. Automatic Modulation Recognition (AMR), as a vitally important technology, not only plays a central role in fields such as cognitive radio, spectrum sensing, and signal monitoring but also has significant implications for improving communication efficiency and the level of intelligence in network management. Nonetheless, as the spectrum environment becomes increasingly complex, especially under adverse conditions such as low signal-to-noise ratio and multipath fading, existing automatic modulation recognition technology faces severe challenges. These challenges include reduced recognition accuracy and insufficient system robustness, urgently necessitating the adoption of new technical solutions.

This thesis introduces and deepens the application of deep learning technology in automatic modulation recognition, proposing a series of innovative solutions aimed at overcoming the limitations of existing technologies and advancing the development of automatic modulation recognition. The contributions of this thesis are mainly focused on the following three aspects:

1. Improving modulation recognition accuracy in low signal-to-noise ratio environments: This thesis proposes an adaptive noise correction module that uses an innovative dynamic adjustment mechanism. This module can automatically adjust its correction strategy based on real-time changes in environmental noise levels, significantly enhancing the system's performance in complex noise backgrounds.

2. Wideband signal modulation recognition: This thesis has made certain advancements in wideband signal modulation recognition, focusing not only on identifying the modulation patterns of signals but also innovatively determining the occupied sub-band positions. This research expands the application range of automatic modulation recognition technology and addresses the new challenges of wideband signal processing.

3. Deep learning-based wideband signal modulation and demodulation: This thesis proposes a comprehensive demodulation framework capable of efficiently extracting key information from wideband signals and performing demodulation. The framework includes identifying sub-band positions, modulation patterns, and symbol lengths, and can output the demodulated bitstream, achieving a complete process from signal detection to information recovery, presenting a complete framework for wideband signal processing.

In summary, by applying deep learning technology, this thesis provides an effective technical approach to solving the challenges faced by automatic modulation recognition technology in complex environments. It holds significant theoretical and practical implications for advancing wireless communication technology and enhancing the level of intelligence in communication networks.

    \KEYWORDS{wireless communication, automatic modulation recognition, deep learning}% 英文关键词
\end{ABSTRACT}
%---------------------------------------------------------------------------%